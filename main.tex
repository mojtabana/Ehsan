\documentclass[12pt,a4paper]{report}

% بسته‌های لازم
\usepackage{fontspec}        % برای فونت فارسی
\usepackage{xepersian}       % برای پشتیبانی کامل از فارسی
\settextfont{XB Niloofar}   % فونت متن (می‌توانی تغییر بدهی)

\usepackage{graphicx}        % برای وارد کردن تصویر
\usepackage{setspace}        % برای فاصله بین خطوط
\onehalfspacing              % فاصله 1.5 بین خطوط

\usepackage{tocloft}         % برای تنظیم فهرست مطالب
\usepackage{geometry}        % تنظیم حاشیه‌ها
\geometry{a4paper, margin=3cm}

\usepackage{hyperref}        % لینک دهی در فهرست
\hypersetup{colorlinks=true, linkcolor=blue}

% اطلاعات پایان‌نامه
\newcommand{\university}{دانشکده حقوق و علوم اجتماعی}
\newcommand{\faculty}{گروه علوم اجتماعی}
\newcommand{\titlefa}{تأثیر زبان آموزش بر وضعیت تحصیل دانش‌آموزان: پژوهشی مردم‌نگارانه در تبریز}
\newcommand{\authorfa}{سولماز صفری ایوقی}
\newcommand{\supervisor}{دکتر اصغر ایزدی جیران}
\newcommand{\advisor}{دکتر محمدباقر علیزاده اقدم}
\newcommand{\year}{1401}

% شروع سند
\begin{document}

% صفحه عنوان
\begin{titlepage}
\centering
{\Large \university \\[0.5cm]}
{\Large \faculty \\[2cm]}
{\Huge \textbf{\titlefa} \\[2cm]}
\begin{tabular}{rl}
  استاد راهنما: & \supervisor \\
  استاد مشاور: & \advisor \\
  دانشجو: & \authorfa \\
\end{tabular}\\[2cm]
{\Large سال: \year}
\end{titlepage}

\pagenumbering{roman}

% تقدیم‌نامه
\chapter*{تقدیم به}
به تمام کسانی که زندگی خود را وقف آموزش و تربیت صحیح نسل جدید کرده‌اند.

% سپاسگزاری
\chapter*{سپاسگزاری}
از استادان ارجمند و دوستانی که در مراحل انجام پژوهش مرا یاری کردند کمال تشکر را دارم.

% چکیده
\chapter*{چکیده}
زبان واقعیتی اجتماعی است که اثرات آن در تمام شئون زندگی انسان قابل مشاهده است... \\
\\
\textbf{کلیدواژه‌ها:} زبان، زبان مادری، آموزش، تحصیل، دانش‌آموز

\newpage

% فهرست مطالب
\tableofcontents
\listoffigures
\listoftables

\newpage
\pagenumbering{arabic}

% فصل اول
\chapter{کلیات تحقیق}
\section{مقدمه}
اینجا متن مقدمه قرار می‌گیرد.

\section{بیان مسئله}
...

\section{اهداف تحقیق}
...

\section{ضرورت تحقیق}
...

% فصل دوم
\chapter{مبانی نظری و پیشینه تجربی}
\section{مبانی نظری}
...

\section{پیشینه تحقیق}
...

% فصل سوم
\chapter{روش تحقیق}
\section{روش گردآوری داده‌ها}
...

\section{روش تحلیل داده‌ها}
...

% فصل چهارم
\chapter{یافته‌ها و تجزیه و تحلیل}
...

% فصل پنجم
\chapter{نتیجه‌گیری و پیشنهادها}
...

% منابع
\chapter*{منابع}
\addcontentsline{toc}{chapter}{منابع}
1. نام نویسنده، عنوان کتاب/مقاله، ناشر، سال انتشار. \\
2. ...

% پیوست‌ها
\appendix
\chapter{پیوست‌ها}
...

\end{document}
